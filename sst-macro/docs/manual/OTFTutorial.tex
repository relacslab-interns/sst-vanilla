%% !TEX root = manual.tex

\section{Using Score-P and OTF2}
\label{sec:tutorial:otf}

OTF2 is part of Score-P. Sources for both can be found here 
\begin{ViFile}
http://www.vi-hps.org/projects/score-p
\end{ViFile}


Trace collection requires both Score-P and OTF2 installations. Trace replay with SST/macro requires only OTF2.


\subsection{Trace Collection}
\label{subsec:otf:traceCollection}

Score-P's default collection strategy will include every function call in the trace, making even small programs produce untenably large traces. Score-P supports collection filters, which can restrict collection at a minimum to MPI and OMP function calls. At the end of the program's runtime, traces from each rank are put in a common directory.  An MPI program must be compiled with Score-P to produce traces:

\begin{ShellCmd}
scorep-mpicxx -o test.exe test.cc
\end{ShellCmd}



To limit the size of the traces, run the program with:

\begin{ViFile}
# these environment variables are picked up by Score-P at runtime
export SCOREP_ENABLE_TRACING=true
export SCOREP_TOTAL_MEMORY=1G
export SCOREP_FILTERING_FILE='scorep.filter'

mpirun -n 2 test.exe
\end{ViFile}

The file \inlineshell{scorep.filter} should contain:
\begin{ViFile}
SCOREP_REGION_NAMES_BEGIN EXCLUDE *
\end{ViFile}


To view a plain-text representation of the trace after running, use the otf2-print tool.

\begin{ViFile}
otf2-print scorep-*/traces.otf2
\end{ViFile}

\subsection{Trace Replay}
\label{subsec:otf:traceReplay}
SST/macro will use a trace replay skeleton for OTF2 in much the same way as it does for dumpi. SST/macro trace replays configured using *.ini files. 

\begin{ViFile}
...

node {
 app1 {
  otf2_timescale = 1.0
  name = otf2_trace_replay_app
  size = N
  otf2_metafile = <trace-root>/scorep-20170309_1421_27095992608015568/traces.otf2
 # debugging output
  otf2_print_mpi_calls=false
  otf2_print_trace_events=false
  otf2_print_time_deltas=false
  otf2_print_unknown_callback=false
 }
}

\end{ViFile}
