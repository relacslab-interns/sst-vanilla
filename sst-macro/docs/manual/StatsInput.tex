%% !TEX root = manual.tex

\section{Statistics}
\label{sec:Statistics}

SST/macro statistics current only work in standalone mode due to the rigid structure of statistics in SST core.
Fixes are planned for SST core for the 10.0 release in 2020.
Until then, most of the statistics will require the standalone core.
Both standalone and SST core do follow the same basic structure, however, and the tutorial below covers concepts important to both.
The three abstractions for understanding statistics are collectors, groups, and outputs.
We will use the running example of a bytes sent statistic.

\subsection{Collectors}
\label{subsec:collectors}

Statistic collectors inherit from \inlinecode{Statistic<T>} and are used by components to collect individual statistics.
Statistics are collected through the \inlinecode{addData} function:

\begin{CppCode}
xmit_bytes_->addData(pkt->byteLength());
\end{CppCode}
The statistic only specifies the \emph{type} of data collected, not the manner of collection (histogram, accumulate).
In this case, e.g., we would have:

\begin{CppCode}
Statistic<uint64_t>* xmit_bytes_;
\end{CppCode}
Statistic objects generally do three things, with the exception of special custom statistics.

\begin{itemize}
\item Collect data through an \emph{addData} function.
\item Register output fields before collecting data through the \inlinecode{registerField} function.
\item Pass along fields to output when done collecting data through the \inlinecode{outputField} function.
\end{itemize}

The data \emph{collected} and the data \emph{output} are not the same - and may not even be of the same type.
For an accumulator, data collected and data output are the same (sum of collected values).
For a histogram, data collected is, e.g., a single integer, while the data output is the counts in a series of bins.
If there are ten bins, the histogram will register ten output fields at the beginning and output ten fields at the end.

\subsection{Outputs}
\label{subsec:outputs}

Statistic outputs provide an abstract interface for outputting fields.
In SQL or Pandas terminology, each statistic defines a row in a table. Each field will be a column.
Outputs therefore map naturally to a CSV file.
SST core also provides an HDF5 output, for those statistics that work with SST core.

\subsection{Groups}
\label{subsec:groups}

Statistics can be grouped together when appropriate, for example bytes sent statistics for switches may want to be grouped together.
Each group is associated with a given output.
Most commonly, a group is only used to link statistics to the same output.
However, aggregation of statistics can potentially be performed as well for certain cases.

\subsection{SST/macro Standalone Input}
\label{subsec:standaloneInput}

Each statistic has a name, which specifies a parameter namespace in the parameter file.
In the case above, we activate an ``xmit\_bytes" statistic.

\begin{ViFile}
nic {
  injection {
   xmit_bytes {
     type = accumulator
     output = csv
     group = test
   }
  }
}
\end{ViFile}
As discussed above, we must specify the type of collection, the type of output, and the group name.
The output file name (test.csv) is generated from the group name. The CSV output looks like:

\begin{ViFile}
name,component,total
xmit_bytes,nid0,16952064
xmit_bytes,nid1,6635264
xmit_bytes,nid2,7542016
\end{ViFile}
For a histogram with 5 bins, we could specify an input:

\begin{ViFile}
 xmit_bytes {
  type = histogram
  output = csv
  group = all
  num_bins = 5
  min_value = 0
  max_value = 5KB
}
\end{ViFile}
which then generates columns for the bins, e.g.

\begin{ViFile}
name,component,numBins,binSize,bin0,bin1,bin2,bin3,bin4
xmit_bytes,nid0,5,1000,1235,559,396,31,3746
xmit_bytes,nid1,5,1000,870,439,322,11,1292
xmit_bytes,nid2,5,1000,838,396,279,11,1551
\end{ViFile}

\subsection{Custom Statistics}\label{subsec:customStats}
Certain statistics (examples below) do not fit into the model of row/column tables and require special \inlinecode{addData} functions.
Rather than declare themselves as \inlinecode{Statistic<T>} for some numeric type T, they declare themselves as \inlinecode{Statistic<void>} and have a completely custom collection and output mechanism.
They also generally do not rely an abstract interfaces. 
For example, a bytes sent statistic output can choose a CSV histogram or an HDF5 accumulator or a text list.
Here, the statistic can only have a single type and works with a specific output.
